\clearpage
\section{Engineering aspects}\label{sec-engineering}

Or ``Build systems in the real world'', or ``Crazy things section''...

\begin{itemize}
    \item Sandboxing
    \item Non-determinism, specifically, its useless form. This doesn't
          include \Excel's \textsf{RANDBETWEEN} function, which is useful.
          To define correctness of non-deterministic build systems, one needs to
          switch from \hs{Applicative} and \hs{Monad} abstractions to
          \hs{Alternative} and \hs{MonadPlus}, respectively, but this is not
          required for actual implementations, most of which can happily accept
          non-deterministic results (with a notable exception of \Buck).
    \item Concurrency: nothing fundamentally difficult and fits our theory, but
          hard to get right in practice.
    \item Cloud aspects: eviction/garbage collection, `frankenbuilds', etc.
    \item Iterative computations, e.g. LaTeX.
    \item Polymorphic keys and values: tasks with multiple outputs, e.g. GHC's
          \cmd{hi} and \cmd{o} files, phony tasks, etc.
\end{itemize}

\subsection{Non-deterministic computations}

Build systems and spreadsheets compute output values from input and intermediate
values. In the most typical case, these \emph{computations} are \emph{functions},
such as \textsf{C1 = A1 + B1}, i.e. their result is uniquely determined by the
input values. However, in general they can be \emph{relations}, i.e. have
multiple valid results. A spreadsheet example: \textsf{A2 = A1 + RANDOM(1,6)}.
This computation has six valid results for each input value \textsf{A1}. In
build systems, the object file \textsf{obj/file.o} is sometimes not uniquely
determined by the source \textsf{src/file.c} -- different compiler runs may
produce different valid results.

Some build systems, e.g. \Buck, require all tasks to be
\emph{deterministic}, i.e. produce exactly the same output when run on the
same inputs. However, not all tasks are deterministic, and there are build
systems that support \emph{non-determinism}. A simple example is \Excel's
\textsf{RANDBETWEEN(low, high)} that returns a random integer in the
interval \textsf{[low, high]}.