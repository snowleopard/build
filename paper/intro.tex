\section{Introduction}\label{sec-intro}

Build systems automate the execution of simple repeatable tasks and are used by
many individuals and organisations. There are software build systems such as
\Make, \Ninja, \Shake, \Buck, \Bazel and many others, as well as various
incremental calculation engines, such as \Excel. Existing build systems vary in
several different ways:

\begin{itemize}
    \item Many build systems, such as the venerable \Make, need to know the
    complete \emph{dependency graph} between tasks at the start of the build
    process. This makes it possible to analyse the graph ahead of time, which
    helps with efficient task scheduling but is fundamentally limited: parts of
    the dependency graph can often be discovered only during the build process,
    i.e. the dependency graph is \emph{dynamic}, not \emph{static}.

    \item When build systems are used by large teams, different team members
    often end up executing exactly the same tasks on their local machines.
    A \emph{cloud build system} can speed up builds dramatically by
    transparently sharing build results among team members. Furthermore, cloud
    build systems allow one to perform \emph{shallow builds} which materialise
    only end build products on a local machine, leaving all the intermediates
    in the cloud. This is a significant optimisation compared to \emph{deep
    builds} that require all transitive dependencies of an end build product to
    be locally available.

    \item Some build systems, e.g. \Buck, require all tasks to be
    \emph{deterministic}, i.e. produce exactly the same output when run on the
    same inputs. However, not all tasks are deterministic, and there are build
    systems that support \emph{non-determinism}. A simple example is \Excel's
    \textsf{RANDBETWEEN(low, high)} that returns a random integer in the
    interval \textsf{[low, high]}.

    \item If a build system executes a task and the resulting value is unchanged
    from the previous run, it can skip the execution of all tasks that depend on
    the value. We call this \emph{early cut-off}. Not all build systems support
    early cut-off: \Make and \Excel do not, whereas \Shake and \Buck do.

    \item Most build systems are designed to track changes of values and rerun
    dependent tasks when they change, but some build systems can also track
    changes in the tasks themselves: if a task is changed, the build system
    will rerun it. For example, when a cell's formula is changed, \Excel will
    recompute its value and propagate the changes. This is uncommon in software
    build systems, where one typically needs to do a full rebuild when changing
    a task.

    \item Most build systems store auxiliary information between builds for
    profiling and optimisation purposes: \Make stores file modification times,
    \Shake stores the discovered dynamic dependency graph, etc.

    % \item Fine-grain dependencies
\end{itemize}

% Build systems solve a complex
% problem and have complex implementations. Different build systems take

...

\todo{AM}{The contribution of this paper is...}

...

Some text copied from the fellowship proposal, not sure whether anything is useful:

A build system is a critical component of most software projects, responsible
for compiling the source code written in various programming languages and
producing executable programs -- end software products. Build systems sound
simple, but they are not; in large software projects the build system grows from
simple beginnings into a huge, complex engineering artefact. Why? Because it
evolves every day as new features are continuously being added by software
developers; because it builds programs for a huge variety of target hardware
configurations (from mobile to cloud); and because it operates under strict
correctness and performance requirements, standing on the critical path between
the development of a new feature and its deployment into production.

It is known that build systems can take up to 27\% of software development
effort, and that improvements to build systems rapidly pay off~\cite{build_maintenance}.
Despite its importance, this subject is severely under-researched, which prompts
major companies, such as Microsoft, Facebook and Google, to invest significant
internal resources to make their own bespoke build system frameworks.

...

Some build systems do not look like build systems but they are. A good example
is spreadsheets, where cells play the role of files, and formulas play the role
of build rules.

...

We separate build systems into \emph{compute} and \emph{build} components,
see~\S\ref{sec-build-abstractions}.

...