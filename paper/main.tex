%% For double-blind review submission, w/o CCS and ACM Reference (max submission space)
\documentclass[acmsmall,review,anonymous]{acmart}\settopmatter{printfolios=true,printccs=false,printacmref=false}
%% For double-blind review submission, w/ CCS and ACM Reference
%\documentclass[acmsmall,review,anonymous]{acmart}\settopmatter{printfolios=true}
%% For single-blind review submission, w/o CCS and ACM Reference (max submission space)
%\documentclass[acmsmall,review]{acmart}\settopmatter{printfolios=true,printccs=false,printacmref=false}
%% For single-blind review submission, w/ CCS and ACM Reference
%\documentclass[acmsmall,review]{acmart}\settopmatter{printfolios=true}
%% For final camera-ready submission, w/ required CCS and ACM Reference
%\documentclass[acmsmall]{acmart}\settopmatter{}

%% Journal information
%% Supplied to authors by publisher for camera-ready submission;
%% use defaults for review submission.
\acmJournal{PACMPL}
\acmVolume{1}
\acmNumber{ICFP} % CONF = POPL or ICFP or OOPSLA
\acmArticle{1}
\acmYear{2018}
\acmMonth{1}
\acmDOI{}
\startPage{1}

%% Copyright information
%% Supplied to authors (based on authors' rights management selection;
%% see authors.acm.org) by publisher for camera-ready submission;
%% use 'none' for review submission.
\setcopyright{none}
%\setcopyright{acmcopyright}
%\setcopyright{acmlicensed}
%\setcopyright{rightsretained}
%\copyrightyear{2018}           %% If different from \acmYear

\bibliographystyle{ACM-Reference-Format}
%% Note: author/year citations are required for papers published as an
%% issue of PACMPL.
\citestyle{acmauthoryear}

%%%%%%%%%%%%%%%%%%%%%%%%%%%%%%%%%%%%%%%%%%%%%%%%%%%%%%%%%%%%%%%%%%%%%%
%% Note: Authors migrating a paper from PACMPL format to traditional
%% SIGPLAN proceedings format must update the '\documentclass' and
%% topmatter commands above; see 'acmart-sigplanproc-template.tex'.
%%%%%%%%%%%%%%%%%%%%%%%%%%%%%%%%%%%%%%%%%%%%%%%%%%%%%%%%%%%%%%%%%%%%%%

\usepackage{bookmark}
\usepackage{booktabs}
\usepackage{subcaption}
\usepackage[utf8]{inputenc}
\usepackage[T1]{fontenc}
\usepackage{xspace}

% Haskell code snippets and useful shortcuts
\usepackage{minted}
\setminted[haskell]{escapeinside=@@}
\newcommand{\hs}{\mintinline{haskell}}
\newcommand{\cmd}[1]{\textsf{\color[rgb]{0,0,0.5} #1}}
\newcommand{\teq}{\smaller $\sim$}
\newcommand{\ghci}{$\lambda$>}
\newcommand{\defeq}{\stackrel{\text{def}}{=}}
\newcommand{\std}[1]{{\color[rgb]{0,0.3,0} #1}}
\newcommand{\blk}[1]{{\color[rgb]{0,0,0} #1}}

% Questions and todos
\newcommand{\q}[2]{\textbf{\color{blue} Question #1:} #2}
\newcommand{\todo}[2]{\textbf{\color{red} #1:} #2}

% Abbreviations for build systems
\newcommand{\Make}{\textsc{Make}\xspace}
\newcommand{\Shake}{\textsc{Shake}\xspace}
\newcommand{\Ninja}{\textsc{Ninja}\xspace}
\newcommand{\Bazel}{\textsc{Bazel}\xspace}
\newcommand{\Buck}{\textsc{Buck}\xspace}
\newcommand{\Excel}{\textsc{Excel}\xspace}
\newcommand{\Calc}{\textsc{Calc}\xspace}

\begin{document}

%% [Short title] Title
\title[Build Systems \`a la Carte]{Build Systems \`a la Carte}
% \titlenote{with title note}
% \subtitle{Subtitle}
% \subtitlenote{with subtitle note}

%% Author information
%% Contents and number of authors suppressed with 'anonymous'.
%% Each author should be introduced by \author, followed by
%% \authornote (optional), \orcid (optional), \affiliation, and
%% \email.
%% An author may have multiple affiliations and/or emails; repeat the
%% appropriate command.
%% Many elements are not rendered, but should be provided for metadata
%% extraction tools.

%% Author with single affiliation.
\author{First1 Last1}
\authornote{with author1 note}          %% \authornote is optional;
                                        %% can be repeated if necessary
\orcid{nnnn-nnnn-nnnn-nnnn}             %% \orcid is optional
\affiliation{
  \position{Position1}
  \department{Department1}              %% \department is recommended
  \institution{Institution1}            %% \institution is required
  \streetaddress{Street1 Address1}
  \city{City1}
  \state{State1}
  \postcode{Post-Code1}
  \country{Country1}                    %% \country is recommended
}
\email{first1.last1@inst1.edu}          %% \email is recommended

%% Author with two affiliations and emails.
\author{First2 Last2}
\authornote{with author2 note}          %% \authornote is optional;
                                        %% can be repeated if necessary
\orcid{nnnn-nnnn-nnnn-nnnn}             %% \orcid is optional
\affiliation{
  \position{Position2a}
  \department{Department2a}             %% \department is recommended
  \institution{Institution2a}           %% \institution is required
  \streetaddress{Street2a Address2a}
  \city{City2a}
  \state{State2a}
  \postcode{Post-Code2a}
  \country{Country2a}                   %% \country is recommended
}
\email{first2.last2@inst2a.com}         %% \email is recommended
\affiliation{
  \position{Position2b}
  \department{Department2b}             %% \department is recommended
  \institution{Institution2b}           %% \institution is required
  \streetaddress{Street3b Address2b}
  \city{City2b}
  \state{State2b}
  \postcode{Post-Code2b}
  \country{Country2b}                   %% \country is recommended
}
\email{first2.last2@inst2b.org}         %% \email is recommended

\begin{abstract}
Build systems are awesome. Build systems are terrifying. In this paper ...
\end{abstract}

%% 2012 ACM Computing Classification System (CSS) concepts
%% Generate at 'http://dl.acm.org/ccs/ccs.cfm'.
\begin{CCSXML}
<ccs2012>
<concept>
<concept_id>10011007.10011006.10011008</concept_id>
<concept_desc>Software and its engineering~General programming languages</concept_desc>
<concept_significance>500</concept_significance>
</concept>
<concept>
<concept_id>10003456.10003457.10003521.10003525</concept_id>
<concept_desc>Social and professional topics~History of programming languages</concept_desc>
<concept_significance>300</concept_significance>
</concept>
</ccs2012>
\end{CCSXML}

\ccsdesc[500]{Software and its engineering~General programming languages}
\ccsdesc[300]{Social and professional topics~History of programming languages}
%% End of generated code

% \keywords{functional programming, build systems}

\maketitle

\section{Introduction}\label{sec-intro}

Build systems are awesome. Build systems are terrifying.

We separate build systems into \emph{compute} and \emph{build} components,
see~\S\ref{sec-compute}.
\clearpage
\section{Background}\label{sec-background}

Build systems automate the execution of simple repeatable tasks for individual
users, as well as for large organisations. There are software build systems,
such as \Make~\cite{feldman1979make}, \Shake~\cite{mitchell2012shake} and
\Bazel~\cite{bazel}, as well as various incremental calculation engines, such
as \Excel~\cite{advanced_excel}. In this section we use these four examples to
introduce main domain-specific notions and requirements. Other notable examples
of build systems and their relation to these four will be discussed
in~\S\ref{sec-related} and~\S\ref{sec-conclusions}.

\subsection{The venerable \Make}

\Make\footnote{There are numerous implementations of \Make and none comes with a
formal specification. In this paper we therefore use a simple and sensible
approximation to a real \Make that you might find on your machine.} was developed
more than 40 years ago to automatically build software libraries and executable
programs from source code. It uses \emph{makefiles} to describe build tasks,
typically referred to as \emph{build rules}, and their dependencies in a simple
text form. For example:

\begin{minted}[frame=single]{makefile}
util.o: util.h util.c
    gcc -c util.c

main.o: util.h main.c
    gcc -c main.c

main.exe: util.o main.o
    gcc util.o main.o -o main.exe
\end{minted}

\noindent
The above makefile lists three tasks: (i) compile a utility library comprising
files \cmd{util.h} and \cmd{util.c} into \cmd{main.o} by
executing\footnote{In this example we treat \cmd{gcc} as a pure function for the
sake of simplicity. In reality there are multiple versions of \cmd{gcc} and the
actual binary that is used to compile and link files is also listed as a task
dependency.} the command \cmd{gcc -c util.c}, (ii) compile the main source file
\cmd{main.c} into \cmd{main.o}, and (iii) link object files \cmd{util.o} and
\cmd{main.o} into the executable \cmd{main.exe}. The makefile contains the
complete information about the task \emph{dependency graph}, which is shown in
Fig.~\ref{fig-make}(a).

\begin{figure}[h]
\begin{subfigure}[b]{0.32\linewidth}
\centerline{\includegraphics[scale=0.28]{fig/make-example.pdf}}
\caption{A dependency graph}
\end{subfigure}
\begin{subfigure}[b]{0.32\linewidth}
\centerline{\includegraphics[scale=0.28]{fig/make-example-full.pdf}}
\caption{A full rebuild}
\end{subfigure}
\begin{subfigure}[b]{0.32\linewidth}
\centerline{\includegraphics[scale=0.28]{fig/make-example-partial.pdf}}
\caption{A partial rebuild}
\end{subfigure}
\caption{A dependency graph and two build scenarios. Input files are shown as
rectangles, intermediate and output files are shown as rounded rectangles. Dirty
inputs and files that are rebuilt are highlighted.
\label{fig-make}}
\end{figure}

If the user modifies the sources of the utility library and runs \Make, it will
perform a full rebuild, because all three tasks transitively depend on the
library, as illustrated in Fig.~\ref{fig-make}(b). On the other hand, if the
user modifies \cmd{main.c} then a partial rebuild is sufficient: indeed, the
file \cmd{util.o} does not need to be rebuilt, since its inputs have not
changed, see Fig.~\ref{fig-make}(c). Files that have changed since the previous
build are called \emph{dirty}.

The requirement to execute tasks \emph{at most once} and only if they
\emph{transitively depend on dirty inputs} is essential for build systems,
their raison d'\^etre. We will call build systems that satisfy this requirement
\emph{minimal}.

To achieve minimality \Make relies on two key ideas. First, it uses \emph{file
modification time} to detect the files that are dirty: a file is marked dirty if
it was modified after the previous build. Second, it constructs a complete task
dependency graph from the information contained in the makefile and executes
tasks in the \emph{topological order}.

...

\subsection{\Excel: a build system in disguise}

...

\Shake

\begin{figure}[h]
\centerline{\includegraphics[scale=0.28]{fig/make-example-cutoff.pdf}}
\caption{An early cut-off example.\label{fig-cutoff}}
\end{figure}



...

\Bazel

...

\begin{itemize}
    \item Many build systems, such as the venerable \Make, need to know the
    complete \emph{dependency graph} between tasks at the start of the build
    process. This makes it possible to analyse the graph ahead of time, which
    helps with efficient task scheduling but is fundamentally limited: parts of
    the dependency graph can often be discovered only during the build process,
    i.e. the dependency graph is \emph{dynamic}, not \emph{static}.

    \item When build systems are used by large teams, different team members
    often end up executing exactly the same tasks on their local machines.
    A \emph{cloud build system} can speed up builds dramatically by
    transparently sharing build results among team members. Furthermore, cloud
    build systems allow one to perform \emph{shallow builds} that materialise
    only end build products on a local machine, leaving all intermediates in the
    cloud. This is a significant optimisation compared to \emph{deep builds}
    that require all transitive dependencies of an end build product to be
    locally available. Non-cloud build systems cannot support shallow builds.

    \item Some build systems, e.g. \Buck, require all tasks to be
    \emph{deterministic}, i.e. produce exactly the same output when run on the
    same inputs. However, not all tasks are deterministic, and there are build
    systems that support \emph{non-determinism}. A simple example is \Excel's
    \textsf{RANDBETWEEN(low, high)} that returns a random integer in the
    interval \textsf{[low, high]}.

    \item If a build system executes a task and the result is unchanged from the
    previous build, it is unnecessary to execute the dependent tasks. We call
    this optimisation \emph{early cut-off}. Not all build systems support early
    cut-off: \Make and \Excel do not, whereas \Shake and \Buck do.

    \item Most build systems track changes of inputs and intermediate results,
    executing dependent tasks whenever they change, but some build systems can
    also track changes in the tasks themselves: if a task has changed, the build
    system will execute it. For example, when a cell's formula has changed,
    \Excel will recompute its value and propagate the changes. We call this
    \emph{self-tracking}. Self-tracking is uncommon in software build systems,
    where one often needs to manually initiate a rebuild of all tasks even if
    just a single task has changed.

    \item Most build systems persistently store auxiliary \emph{build
    information} for profiling and optimisation purposes: \Make stores file
    modification times, \Shake stores the discovered dynamic dependency graph,
    \Bazel and other cloud build systems store information about inputs and
    outputs of previously executed tasks, etc.
\end{itemize}

This paper presents a purely functional abstraction for build systems that
allows us to express all the above intricacies of build systems and design
complex build systems from simple primitives. The presented abstraction fits in
just two lines of Haskell code, which are explained
in~\S\ref{sec-abstractions}:

\begin{minted}{haskell}
type Compute c k v = @\std{forall}@ f. c f => (k -> f v) -> k -> Maybe (f v)
type Build c i k v = Compute c k v -> k -> Maybe i -> Map k v -> (i, Map k v)
\end{minted}


Below we define basic notions used in build systems and other similar domains,
for example, spreadsheets.

\subsection{Keys, values, hashes and store}

\emph{Keys} are used to distinguish \emph{values}. In build systems keys are
typically filenames, e.g. \textsf{src/file.c}, whereas values are file contents
(a C program source code in this case). In spreadsheets keys are cell names,
e.g. \textsf{A1}, and values are numbers, text, etc. that are typically displayed
inside cells. We will use type variables \hs{k} and \hs{v} to denote keys and
values, respectively.

A \emph{store} associates keys to values. It is convenient to assume that a store
is total, i.e. it contains a value for every possible key. We therefore also
assume that the type of values is capable of encoding values corresponding to
non-existent keys (missing files or empty cells).

We use a cryptographic \emph{hash function} \hs{hash :: v -> Hash} for
efficient tracking and sharing of build results.

\subsection{Input, intermediate and output values}

Some values must be provided by the user as \emph{input}. For example,
\textsf{src/file.c} can be edited by the user who relies on the build system to
compile it into \textsf{obj/file.o}. Similarly, the user can input \textsf{A1 = 5}
and \textsf{B1 = 9} expecting the spreadsheet to compute their sum in \textsf{C1},
i.e. \textsf{C1 = 14}.

In the above examples, \textsf{obj/file.o} and \textsf{C1} are \emph{output} values.

In some situations we might also need the notion of \emph{intermediate} values,
which are not interesting for the user but are produced in the process of turning
inputs into outputs. For example, the user might only be interested in the
executable \textsf{bin/file.exe} obtained by linking \textsf{obj/file.o} with
standard libraries, in which case \textsf{obj/file.o} can be considered an
intermediate value.

\subsection{Non-deterministic computations}

Build systems and spreadsheets compute output values from input and intermediate
values. In the most typical case, these \emph{computations} are \emph{functions},
such as \textsf{C1 = A1 + B1}, i.e. their result is uniquely determined by the
input values. However, in general they can be \emph{relations}, i.e. have
multiple valid results. A spreadsheet example: \textsf{A2 = A1 + RANDOM(1,6)}.
This computation has six valid results for each input value \textsf{A1}. In
build systems, the object file \textsf{obj/file.o} is sometimes not uniquely
determined by the source \textsf{src/file.c} -- different compiler runs may
produce different valid results.

\subsection{Dynamic dependencies}

...

\todo{AM}{Add two examples: include files and cyclic spreadsheet computations.}

\subsection{Requirements for build systems}

\begin{itemize}
    \item Correctness
    \item Minimality
    \item Support for sharing and skipping intermediate values
\end{itemize}
...

\section{Compute}\label{sec-compute}

\begin{figure}
\begin{minted}{haskell}
type IdentityCompute    k v = forall f.                  (k -> f v) -> k -> f v
type FunctorialCompute  k v = forall f. Functor     f => (k -> f v) -> k -> f v
type ApplicativeCompute k v = forall f. Applicative f => (k -> f v) -> k -> f v
type AlternativeCompute k v = forall f. Alternative f => (k -> f v) -> k -> f v
type MonadicCompute     k v = forall m. Monad       m => (k -> m v) -> k -> m v
type MonadPlusedCompute k v = forall m. MonadPlus   m => (k -> m v) -> k -> m v
\end{minted}
\caption{Compute abstractions}\label{fig-compute}
\end{figure}

Consider abstractions in Fig.~\ref{fig-compute}.


\section{Build systems, abstractly}\label{sec-build}

In this section we further analyse the \hs{Build i c k v} abstraction, in
particular, we analyse two main aspects of every build system: the metadata
\hs{i} it persistently stores between builds, and the algorithm it uses to
compute values in the dependency order.

\subsection{Metadata: dirty bit, dependency graph, cache}\label{sec-build-metadata}

...

\subsection{Algorithm: topological, reordering, recursive}\label{sec-build-algorithm}

...

\clearpage
\section{Build systems, concretely}\label{sec-examples}

In the previous sections we've discussed the types of build system in text, but the way we came to our conclusions was by developing a framework to fit the builds in. In this section we explain the framework, some of the abstractions we introduced (which model the properties from \S\ref{todo}), and how we can make a meaningful implementation of the build systems discussed in \S\ref{sec-background}.

\subsection{Framework}

As we saw in the introduction, a build system can be defined as:

\begin{minted}{haskell}
type Build c i k v = Compute c k v -> k -> Maybe i -> Map.Map k v -> (i, Map.Map k v)
\end{minted}

Using the framework we have built we can express the simplest and stupidest build system as:

\begin{minted}{haskell}
dumb :: Build Monad () k v
dumb compute = runM . f
    where f k = maybe (getStore k) (putStore k =<<) $ compute f k
\end{minted}

\begin{figure}
\begin{minted}{haskell}
data M i k v r
runM :: Default i => M i k v a -> Maybe i -> Map.Map k v -> (i, Map.Map k v)
getStore :: k -> M i k v v
putStore :: k -> v -> M i k v v
\end{minted}
Note: we have omitted the context on \textt{k}, things like \texttt{Ord} (so the implementation can use a finite map) or \texttt{Show} (so the implementation can produce nice error messages).
\caption{API with which to implement build systems}
\label{fig-M-api}
\end{figure}
  
The types of important definitions are given in Figure \ref{fig-M-api}





but 


This section present several concrete examples of build systems, providing
simple implementations that use the previously introduced abstractions.

\subsection{Applicative build systems}

Applicative build system can only accept an applicative compute, because they
rely on building the full dependency graph upfront, which is impossible with
dynamic dependencies.

\vspace{4mm}
\subsubsection{An incorrect applicative build system}~\\

\todo{AM}{Describe the \hs{dumbBuild} that builds outputs in the given order.}

\vspace{4mm}
\subsubsection{\Make: a correct but non-minimal applicative build system}~\\
 
We start by applicative build systems, such as the classic vanilla
\Make\footnote{There are numerous implementations of \Make and none comes with a
formal specification. In this paper we therefore use a simple but reasonable
approximation to a real \Make that you might find on your machine.} and a more
recent \Ninja.

Make is correct but non-minimal. It is unusual from other build systems in that
it relies on \emph{timed values}, i.e. on a map \hs{age :: v -> Time}, as a
heuristic to decide whether a value is up-to-date.

\todo{NM}{Add a description of \Make.}

\vspace{4mm}
\subsubsection{\Ninja: a correct and minimal applicative build system}~\\

\Ninja is a modern alternative to \Make ...

\todo{NM}{Add a description of \Ninja.}

\todo{AM}{What about Nix? John Ericson suggested in a blog comment that it may
be somewhat monadic, see:
\url{https://blogs.ncl.ac.uk/andreymokhov/cloud-and-dynamic-builds/\#comment-1849}.}

\subsection{Monadic build systems}

...
\subsubsection{An incorrect monadic build system}~\\

See \hs{dumbBuild}.

\subsubsection{Correct but not minimal monadic build system}~\\

As of writing, XXX is a correct but non-minimal monadic build system. Here is
an example where it does unnecessary computation:

\begin{minted}[frame=single]{text}
A1 = 10
A2 = MIN(A1, 4)
A3 = SQRT(A2)
\end{minted}

After performing the build, XXX computes \textsf{A2 = 4} and \textsf{A3 = 2}
from the input cell \textsf{A1}. Now if the user changes the \textsf{A1} to 5,
XXX will recompute both \textsf{A2 = 4}, which is necessary, but also
\textsf{A3 = 2}, which is unnecessary since its only dependency (\textsf{A1})
has not changed since the previous build.

One possible implementation of a spreadsheet build
system\footnote{For example, see \url{http://www.decisionmodels.com/calcsecretsc.htm}.}
is to maintain a
sequence of cells in which they should be evaluated. The build system takes this
sequence as input, marks all cells as \emph{unevaluated}, and then attempts to
evaluate all cells in the order specified by the sequence. If the cell to be
evaluated depends on an unevaluated cell, this means the provided sequence is
not a correct topological order and is updated by moving the current cell to the
back of the sequence. The build then proceeds with the next cell. Otherwise, if
all dependencies of the current cell have been marked as \emph{evaluated}, the
build system computes the correct value for the cell by evaluating the formula,
writes it into the spreadsheet and proceeds to the next cell in the sequence. If
there are no cyclic dependencies, this process eventually terminates with
correct results. The resulting sequence, which is guaranteed to respect the
topological order of dependencies is stored to be reused during the next build.

...

Interesting note: XXX is unusual in that it tracks changes in the compute,
i.e. if a user edits a formula, XXX will correctly rebuild results. Most build
systems, including \Shake, do not meet this requirement and a clean rebuild may
be necessary if the user edits build rules.

...

\subsubsection{Correct and minimal monadic build system}~\\

\Shake is a correct and minimal monadic build system.
\todo{NM}{Describe \Shake using our abstractions.}

\subsection{Cloud build systems}

Codebase of large software projects comprises millions of lines of code spread
across thousands of files, each built independently by thousands of developers
on thousands of machines. A distributed cloud build system speeds up builds
dramatically (and saves energy) by transparently sharing build products among
developers.

\todo{AM}{Also mention YYY (developed by ZZZ), Buck (Facebook) and Pants
(Twitter et al). Buck is essentially the same. Are they essentially the same?}

\Bazel is a cloud build system developed by Google, which supports caching of
build results. To achieve that, it maintains two partial maps:

\begin{minted}{haskell}
type Cache =  Hash  -> Maybe v
type Known = [Hash] -> Maybe Hash
\end{minted}

\hs{Cache} is a conventional \emph{content-addressable store}, that can be used
to fetch a previously computed value given its hash.

\hs{Known} records a known outcome of a computation that took a list of values
as input (represented by their hashes) and produced a resulting value (also
represented by its hash). Now, a build system has the following two options to
recompute a value:

\begin{itemize}
    \item Call the compute function on the store containing all up-to-date
    dependencies and put the computed value to the store.
    \item Collect hashes of all up-to-date dependencies \hs{hs} and query the
    \hs{Known} map. If this computation has been performed before, the map will
    contain the hash \hs{h} of one valid result (recall that the compute function
    may be non-deterministic). It is now possible to lookup \hs{h} in the
    \hs{Cache} and if it contains a value \hs{v} it can be downloaded to the
    local store instead of running the compute.
\end{itemize}

...

\todo{AM}{Also describe the XYZ algorithm.}

\clearpage
\section{Engineering aspects}\label{sec-engineering}

Or ``Build systems in the real world'', or ``Crazy things section''...

\subsection{Partial stores, exceptions}

Given a \hs{Build c i k v} defined for a total store, one can use it
with a partial store \hs{k -> Maybe v}, obtaining a special case
\hs{Build c i k (Maybe v)}, or with a store \hs{k -> Either e v} that can
throw exceptions of type \hs{e}, obtaining a special case
\hs{Build c i k (Either e v)}.

\subsection{Non-deterministic computations}\label{sec-non-determinism}

Sandboxing: guarding against non-determinism due to missing dependencies.

\todo{AM}{Rewrite everything below.}

Non-determinism, specifically, its \emph{useless} form. This doesn't
include \Excel's \textsf{RANDBETWEEN} function, which is useful.
To define correctness of non-deterministic build systems, one needs to
switch from \hs{Applicative} and \hs{Monad} abstractions to
\hs{Alternative} and \hs{MonadPlus}, respectively, but this is not
required for actual implementations, most of which can happily accept
non-deterministic results (with a notable exception of \Buck).

... Some old text:

Build systems and spreadsheets compute output values from input and intermediate
values. In the most typical case, these \emph{computations} are \emph{functions},
such as \textsf{C1 = A1 + B1}, i.e. their result is uniquely determined by the
input values. However, in general they can be \emph{relations}, i.e. have
multiple valid results. A spreadsheet example: \textsf{A2 = A1 + RANDOM(1,6)}.
This computation has six valid results for each input value \textsf{A1}. In
build systems, the object file \textsf{obj/file.o} is sometimes not uniquely
determined by the source \textsf{src/file.c} -- different compiler runs may
produce different valid results.

Some build systems, e.g. \Buck, require all tasks to be
\emph{deterministic}, i.e. produce exactly the same output when run on the
same inputs. However, not all tasks are deterministic, and there are build
systems that support \emph{non-determinism}. A simple example is \Excel's
\textsf{RANDBETWEEN(low, high)} that returns a random integer in the
interval \textsf{[low, high]}.

\subsection{Concurrency}\label{sec-concurrency}

Nothing fundamentally difficult and fits our theory, but hard to get right in
practice.

\todo{NM}{Add a paragraph on how to introduce concurrency to implementations
in~\S\ref{sec-implementations}}.

\subsection{Cloud aspects}\label{sec-cloud-aspects}

Cloud aspects: eviction/garbage collection, `frankenbuilds', etc.

\todo{AM}{Descibe frankenbuilds linking to CloudBuild paper.}

\subsection{Tracking and self-tracking}\label{sec-tracking-aspects}

Annotating dependency graphs with hashes, times, etc. is in many cases just an
engineering trade-off.

\todo{NM}{Clarify, e.g. Shake's \cmd{ChangeModtime}.}

\todo{NM}{Explain how self-tracking can be implemented.}

\subsection{Iterative computations}\label{sec-iterative-compute}

Iterative computations, e.g. LaTeX. \Excel has built-in support for this.

\todo{NM}{Clarify how this fits with the example implementations.}

\subsection{Polymorphism}\label{sec-polymorphism}

Polymorphic keys and values: tasks with multiple outputs, e.g. GHC's \cmd{hi}
and \cmd{o} files, phony tasks, etc.

\todo{AM}{Give an example.}

\todo{NM}{Describe how to add this feature to \Shake.}

\clearpage
\section{Related work}\label{sec-related}

\begin{itemize}
    \item Memoization
    \item Self-adjusting computation
    \item Lens-like formulation
\end{itemize}

\input{conclusions}

%% Acknowledgments
\begin{acks}
  %% acks environment is optional
  %% contents suppressed with 'anonymous'
  %% Commands \grantsponsor{<sponsorID>}{<name>}{<url>} and
  %% \grantnum[<url>]{<sponsorID>}{<number>} should be used to
  %% acknowledge financial support and will be used by metadata
  %% extraction tools.
  We would like to thank ...
\end{acks}

\bibliography{refs}

%% Appendix
% \appendix
% \section{Appendix}
% Text of appendix \ldots

\end{document}
