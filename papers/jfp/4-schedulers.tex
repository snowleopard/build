\section{Build Systems \`a la Carte Schedulers}\label{sec-build}

\todo{Elaborate (it's very terse right now) with examples}

The focus of this paper is on a variety of implementations of
\hs{Build}~\hs{c}~\hs{i}~\hs{k}~\hs{v}, given
a \emph{client-supplied} implementation of \hs{Tasks}~\hs{c}~\hs{k}~\hs{v}. That
is, we are going to take \hs{Tasks} as given from now on, and explore variants of
\hs{Build}: first abstractly (in this section) and then concretely
in~\S\ref{sec-implementations}.

As per the definition of minimality~\ref{def-minimal}, a minimal build
system must \textbf{rebuild only out-of-date keys} and at most once. The only
way to achieve the ``at most once'' requirement while producing a correct build
result (\S\ref{sec-build-correctness}) is to \textbf{build all keys in an
order that respects their dependencies}.

\vspace{1mm}
We have emboldened two different aspects above: the part of the
build system responsible for scheduling tasks in the dependency order
(a `scheduler') can be cleanly separated from the part responsible for deciding
whether a key needs to be rebuilt (a `rebuilder'). We tackle each
aspect separately in subsections~\S\ref{sec-dependency-orderings}
and~\S\ref{sec-out-of-date}.

\subsection{The Scheduler: Respecting the Dependency Order}
\label{sec-dependency-orderings}

Section \S\ref{sec-background} introduced three different \emph{task schedulers}
that decide which tasks to execute and in what order; see the ``Scheduler'' column
of Table~\ref{tab-summary} in \S\ref{sec-background-summary}.
This subsection explores the properties of the three schedulers, and
possible implementations.

\vspace{-2mm}
\subsubsection{Topological}\label{sec-topological}

The topological scheduler pre-computes a linear order of tasks, which when followed,
ensures the build result is correct regardless of the initial store. Given a
task description and the output \hs{key}, you can compute the linear order by
first finding the (acyclic) graph of the \hs{key}'s reachable dependencies, and
then computing a topological sort. However, as we have seen in~\S\ref{sec-deps},
we can only extract dependencies from an applicative task, which requires the
build system to choose \hs{c}~\hs{=}~\hs{Applicative}, ruling out dynamic
dependencies.

\vspace{-2mm}
\subsubsection{Restarting}\label{sec-restarting}

To handle dynamic dependencies we can use the following approach: build tasks in
an arbitrary initial order, discovering their dependencies on the fly; whenever
a task calls \hs{fetch} on an out-of-date key \hs{dep}, abort the task, and
switch to building the dependency \hs{dep}; eventually the previously aborted
task is restarted and makes further progress thanks to \hs{dep} now being up to
date. This approach requires a way to abort tasks that have failed due to
out-of-date dependencies. It is also not minimal in the sense that a task may
start, do some meaningful work, and then abort.

To reduce the number of aborts (often to zero) \Excel records the
discovered task order in its \emph{calc chain}, and uses it as the
starting point for the next build (\S\ref{sec-background-excel}).
\Bazel's restarting scheduler does not store the discovered order
between build runs; instead, it stores the most recent task dependency
information. Since this information may become outdated, \Bazel may
also need to abort a task if a newly discovered dependency is out of date.

\vspace{-2mm}
\subsubsection{Suspending}\label{sec-suspending}

An alternative approach, utilised by the \hs{busy} build system
(\S\ref{sec-general-build}) and \Shake, is to simply build dependencies when
they are requested, suspending the currently running task. By combining that
with tracking the keys that have already been built, one can obtain a minimal
build system with dynamic dependencies.

This approach requires that a task may be started and then suspended until
another task is complete. Suspending can be done with cheap green threads and
blocking (the original approach of \Shake) or using continuation-passing style
\cite{claessen_continuations} (what \Shake currently does).
